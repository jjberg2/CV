\documentclass[letterpaper]{article}

\usepackage{hyperref}
\usepackage{geometry}
\usepackage{etaremune}
%\usepackage{eurofont}

% Comment the following lines to use the default Computer Modern font
% instead of the Palatino font provided by the mathpazo package.
% Remove the 'osf' bit if you don't like the old style figures.
\usepackage[T1]{fontenc}
\usepackage[sc,osf]{mathpazo}

% Set your name here
\def\name{Jeremy Berg}

% Replace this with a link to your CV if you like, or set it empty
% (as in \def\footerlink{}) to remove the link in the footer:
\def\footerlink{}

% The following metadata will show up in the PDF properties
\hypersetup{
  colorlinks = true,
  urlcolor = black,
  pdfauthor = {\name},
  pdfkeywords = {economics, statistics, mathematics},
  pdftitle = {\name: Curriculum Vitae},
  pdfsubject = {Curriculum Vitae},
  pdfpagemode = UseNone
}

\geometry{
  body={6.5in, 9in},
  left=1.0in,
  top=1.0in
}

% Customize page headers
\pagestyle{myheadings}
\markright{\name}
\thispagestyle{empty}

% Custom section fonts
\usepackage{sectsty}
\sectionfont{\rmfamily\mdseries\Large}
\subsectionfont{\rmfamily\mdseries\itshape\large}

% Other possible font commands include:
% \ttfamily for teletype,
% \sffamily for sans serif,
% \bfseries for bold,
% \scshape for small caps,
% \normalsize, \large, \Large, \LARGE sizes.

% Don't indent paragraphs.
\setlength\parindent{0em}

% Make lists without bullets
\renewenvironment{itemize}{
  \begin{list}{}{
    \setlength{\leftmargin}{1.5em}
  }
}{
  \end{list}
}

\begin{document}

% Place name at left
{\huge \name}

% Alternatively, print name centered and bold:
%\centerline{\huge \bf \name}

\vspace{0.25in}

%ADDRESS
\begin{minipage}{0.45\linewidth}
  \href{http://www.ucdavis.edu/}{University of California - Davis} \\
  Center for Population Biology \\
  Davis, CA 95616
\end{minipage}
\begin{minipage}{0.45\linewidth}
  \begin{tabular}{ll}
    Phone: & (715) 316-3815 \\
%    Fax: &  (530) 752-4604 \\
    Email: & \href{jjberg@ucdavis.edu}{\tt jjberg@ucdavis.edu} \\
%    Web: & \href{http://www.rilab.org/}{\tt www.rilab.org} \\
  \end{tabular}
\end{minipage}

%EDUCATION
\section*{Education}
\begin{itemize}
 \item BSc Biology - Evolution, University of Wisconsin - Madison  2010
 \item PhD Population Biology, University of California - Davis  2016
% \item PhD Genetics, University of Georgia 2006
 \end{itemize}

%EMPLOYMENT
%\section*{Employment}
%\begin{itemize}
%
%\item Assistant Professor, Dept. Plant Sciences, University of California - Davis 2009-present
%\item Postdoctoral Researcher, University of California - Irvine 2006-2008
%\item Profesor de Asignatura, Universidad Nacional Aut\'{o}noma de M\'{e}xico 2001
%\item Profesor de Inglés, Boston Language Institute, Ciudad de México 2000-2001 
%\item Field Botanist, University of California Riverside Herbarium 1998
%
%\end{itemize}

%CURRENT FUNDING
%\section*{Current Funding}
%\begin{itemize}
%\item USDA-NIFA Plant Genome, Genetics, and Breeding: ''Scanning for yield: high-throughput discovery of candidate agronomic loci for marker-assisted selection in maize'' (PI, \$448,000) Sept 2009 - Sept 2012
%\item NSF Plant Genome Research Program: ''GEPR: Functional Genomics of Maize Centromeres'' (Co-PI, \$754,409) July 2010 - Aug 2015 
%\item USDA-ARS: ''Collection of the maize wild relative, Zea luxurians, in southeast Guatemala for ex situ conservation'' (PI, \$9,000) Nov 2010 - June 2012
%\item UC MEXUS: ''Phylogeography and Systematics of Mesoamerican \emph{Diospyros}''  (PI, \$14,825) July 2009 - Jan 2012
%\end{itemize}

%AWARDS
\section*{Selected fellowships and awards}
\begin {itemize}

\item NSF Graduate Research Fellowship Program 2013
\item SMBE Post-Graduate Travel Award 2013
%\item Dissertation Completion Fellowship, University of Georgia 2005-2006
%\item Sigma Xi Grants in Aid of Research 2004
%\item NIH Training Grant, predoctoral research assistantship 2003-2005
%\item University-wide Fellowship, University of Georgia 2001-2003
%\item Chancellor's Distinguished Fellowship, UC Riverside 1998-2000
%\item National Science Foundation Research Experience for Undergraduates 1997 
%\item Student of the Year, College of Natural and Agricultural Sciences, UC Riverside 1998 
%\item University of California Regents Honorarium 1995-1998
%\item Academic Senate Travel Award, UC Davis 2010
%\item Center for Latin American and Caribbean Studies, Travel Grant 2003
%\item University of California Germplasm Research Center Grant 2000
%\item Myron Winslow Scholarship 1997-1998 
%\item Bedding Plants Society Int, John Rathemore Memorial Scholarship 1999-2000
%\item University of California MEXUS Travel Grant 1999

\end{itemize}

%ADVISING
%\section*{Advising}
%\begin{itemize}

%\item Postdoctoral Scholars: Matthew Hufford, Tanja Pyh\"aj\"arvi
%\item PhD Students: Paul Bilinksi (Plant Biology), Laura Vann (Genetics), Dianne Velasco (Genetics)
%\item undergraduate students: PuiYan Ho, Thomas Kono
%\item PhD Dissertation Committee: Allen Kovach (Genetics), Jason Corwin (Plant Biology), Chad Jorgensen (Hort \& Agronomy)
%\item MS Thesis Committee: Joanne Heraty (International Ag. & Development)
%\item Former advisees: Joost van Heerwaarden (postdoc), Nikhil Ghopal, Lauren Sagara, and Casper Thommes (undergrad)
%\end{itemize}

%PROFESSIONAL SERVICE
\section*{Professional Service}
\begin{itemize}

%\item University of California Mexico/US grant review panel, 2011
%\item Scientific Advisory Board, AMAIZING Project (\euro 30 million to INRA), 2011-2018 
%\item Chair, Dept. of Plant Sciences IT committee
%\item NSF-USDA Phenomics workshop, 2011
%\item NIFA Fellowship review panel, USDA, 2011
%\item College committee on strategic planning for plant breeding, 2011
%\item Dept. of Plant Sciences academic planning committee, 2010-2011
%\item Ad hoc reviewer, NSF, 2010
%\item Evolutionary genetics seminar chair, Genetics Graduate Group, 2010-present
%\item Associate editor, American Journal of Botany 2009-present
%\item Executive Committee, Genetics Graduate Group, UC Davis 2009-present
%\item USDA-DOE Grant Review Panel, 2010
%\item Minnesota Agricultural Experiment Station, Hatch Project external reviewer, 2010
%\item Grant review 2010: NSF, USDA panel, Minn. Ag. Experiment Station
%\item Journal peer review in 2010: Nat. Genetics, PLoS Genetics, PNAS, Genetics, MBE, Heredity, AJB, PLoS ONE
%\item Chair, Plant Breeding and Biodiversity, Genetics Graduate Group, UC Davis 2009-2010
%\item Seminar Committee, Plant Biology Graduate Group, UC Davis, 2009, 2010
\item Recent journal peer review: eLife, PLOS Genetics, MBE, Evolution, Human Genetics

\end{itemize}

%INSTRUCTION & OUTREACH
\section*{Instruction and Outreach}
\begin{itemize}

\item Invited Lecturer: 3 day graduate course in ``Population Structure and the Genetic Architecture of Quantitative Traits'' - Uppsala University, Sweden October 2013
\item Teaching Assistant: BIS101 Genes and Gene Regulation, UC Davis, Winter 2013
 \item Moderator: The AskScience Forum: askscience.reddit.com Spring 2011 - present
%\item US-Mexico Functional Genetics of Maize Centromeres student exchange program, 2011-present
%\item Population and Quantitative Genetics, GGG 201D, UC Davis, Winter 2010-present (5 units, $\frac{1}{2}$ teaching)
%	\subitem Winter 2010, 17 students, instructor rating 4.2, course rating 3.7
%	\subitem Winter 2011, 23 students, instructor rating 4.1, course rating 3.7
%\item Plant Genetics, PLS 152, UC Davis, Fall 2011-present (4 units, $\frac{1}{2}$ teaching)
%	\subitem Fall 2010, 31 students, instructor rating 4.2, course rating 4.0
%\item Guest lecturer, Ethnobotany, Winter 2011
%\item Guest lecturer, Plant Biology Core Course, guest lecturer, Fall 2010
%		\subitem Fall 2010 (2 lectures), 9 students, instructor rating 4.4, course rating 3.8
%\item Seminar in plant breeding and biodiversity, GGG 292, UC Davis 2009 (2 units)
%		\subitem Fall 2009, 10 graduate students, instructor rating 4.5, course rating 4.5
%\item Guest lecturer, College Success Institute, UCD Academic Preparation Programs 2009
%\item Open-source Perl, C++, Bash software distribution, 2007-present
%\item Volunteer instructor, NSF Science Behind Our Food Program 2005
%\item Graduate Mentor, Society for the Study of Evolution Diversity Program 2004
%\item Invited guest lecturer, Genetics, U. Georgia 2003-2005
%\item Graduate Student Teaching Intern for Evolution, U. Georgia 2004
%\item Biolog\'{i}a de Plantas I, UNAM 2001
%\item Teaching Assistant for Ethnobotany, UC Riverside 1999
%\item Teaching Assistant for Genetics, U. Georgia 2003


\end{itemize}

%SEMINARS
%\section*{Recent Invited Seminars}
%\begin{itemize}
%\item ASA/CSSA/SSSA Convention, Symposium on maize biology, San Antonio, Oct. 2011 
%\item Dept. of Plant \& Microbial Biology, UC Berkeley, Sept. 2011 
%\item Seminis Vegetable Seeds, Woodland CA, 2011 
%\item Dept. of Plant Sciences, UC Davis 2011
%\item Dept. of Botany and Plant Sciences, UC Riverside 2011
%\item USDA Agricultural Research Service, Iowa State 2010
%\item Microbial and Plant Genomics Institute, U. Minnesota 2010
%\item Society for Molecular Biology and Evolution, Plant Ecological Genomics Symposium, Lyon 2010
%\item Dept. of Plant Sciences, UC Davis 2009
%\item Instituto de Ecolog\'{i}a, Universidad Nacional Aut\'{o}noma de M\'{e}xico 2008
%\item Harlan II Symposium, UC Davis 2008
%\item Dept. of Biology, UC Riverside 2008
%\item Secretar\'{i}a de Medio Ambiente y Recursos Naturales, GMO Risk Assessment Symposium, Mexico City 2008
%\item Dept. of Plant Sciences, UC Davis 2007
%\item Dept. of Biology, York University 2007
%\item Dept. of Botany and Plant Sciences, UC Riverside 2007
%\item Georgia Partnership for Reform in Science and Mathematics (PRISM), U. Georgia 2004
%\item University of Georgia Chapter of Sigma-Xi, U Georgia 2004

%\end{itemize}

%CONF. STUFF
\section*{Conferences}
\begin{itemize}
\item Evolution 2016 (1 abstract)
 \item Probabilistic Modeling in Genomics 2015 (1 abstract)
 \item SMBE 2015 (1 abstract)
 \item Evolution 2014 (1 abstract)
 \item SMBE 2013 (2 abstracts)
 \item Evolution 2013 (2 abstracts)
%\item Maize Genetics 2010 (4 abstracts)
%\item Society for Molecular Biology and Evolution 2010 (1 abstract)
\end{itemize}

%PUBS
\section*{Publications}
%SUBMITTED
%\subsection*{Submitted}
%\begin{itemize}
%
%
%
%\item Hufford, MB, Xun X, van Heerwaarden J, Pyh\"aj\"arvi T, Chia J-M, Cartwright RA, Elshire RJ, Glaubitz JC, Guill KE, Kaeppler S, Lai J, Morrell PL, Shannon LM, Song C, Spinger NM, Swanson-Wagner RA, Tiffin P, Wang J, Zhang G, Doebley J, McMullen MD, Ware D, Buckler ES, Yang S, {\bf Ross-Ibarra J}. Population genomics of domestication and improvement in maize.
%\item  Chia, J-M, Song C, Bradbury P, Birchler, JA, Costich D, de Leon N, Doebley JC, Elshire RJ, Geller L, Glaubitz JC, Gaut BS, Gore M, Guill KE, Holland J,  Lai J, Li M, Liu X, Lu Y, McCombie R, Prasanna BM, Rong T, Sekhon R, Tenaillon M, Tian F, Sun Q, Wang J, Xu X, Zhang Z, Kaeppler S, {\bf Ross-Ibarra J}, McMullen M, Buckler ES, Zhang G, Xu Y, Ware, D.  Capturing extant variation from a genome in flux: maize HapMap II.
%
%\item Cook, JP, McMullen JD, Holland JB, Tian F, Bradbury P, {\bf Ross-Ibarra J}, Buckler ES, Flint-Garcia SA. Genetic architecture of maize kernel composition in the Nested Association Mapping and Inbred Association panels.
%\item  van Heerwaarden J, Hufford MB, {\bf Ross-Ibarra J}. Historical patterns of genomic ancestry and selection in North American maize.
%\end{itemize}

%PRINT/PRESS
%\subsection*{In press or in print}
\begin{etaremune}
\item {\bf Berg JJ}, Coop G. (2015)  A Coalescent Model for a Sweep of a Unique Standing Variant Genetics 201: 707-725 
\item {\bf Berg JJ}, Coop G. (2014)  A Population Genetic Signal of Polygenic Adaptation PLOS Genetics 10: e1004412
\item Martins TR,  {\bf Berg JJ}, Blinka S, Rausher MD, Baum DA. (2013)  Precise spatio-temporal regulation of the anthocyanin biosynthetic pathway leads to petal spot formation in Clarkia gracilis (Onagraceae) New Phytologist 192: 958-969

%\item van Heerwaarden J, Doebley J, Briggs WH, Glaubitz JC, Goodman MM, S\'{a}nchez Gonz\'{a}lez JJ, {\bf Ross-Ibarra J}. (2011). Genetic signals of origin, spread and introgression in a large sample of maize landraces. PNAS 108: 1088-1092
%\item Studer, A, Zhao Q, {\bf Ross-Ibarra J}, Doebley J.  (2011) Identification of a functional transposon insertion in the maize domestication gene \emph{tb1}. Nature Genetics \emph{In Press}.
%\item Tenaillon, MI, Hufford MB, Gaut BS, {\bf Ross-Ibarra J}. (2011)  Genome size and TE content as determined by high-throughput sequencing in maize and \emph{Zea luxurians}. Genome Biology and Evolution  3: 29-229
%\item van Heerwaarden J, Doebley J, Briggs WH, Glaubitz JC, Goodman MM, S\'{a}nchez Gonz\'{a}lez JJ, {\bf Ross-Ibarra J}. (2011). Genetic signals of origin, spread and introgression in a large sample of maize landraces. PNAS 108: 1088-1092
%\item Hufford MB, Gepts P, {\bf Ross-Ibarra J}.  (2011) Influence of cryptic population structure on observed mating patterns in the wild progenitor of maize (Zea mays ssp. parviglumis). Molecular Ecology 20: 46-55
%\item Eckert, AJ, van Heerwaarden J, Wegrzyn JL, Nelson CD, {\bf Ross-Ibarra J}, Gonz\'{a}lez-Mart\'{i}nez SC, and Neale DB (2010). Patterns of population structure and environmental associations to aridity across the range of loblolly pine (\emph{Pinus taeda} L, Pinaceae). Genetics 185: 969-982
%\item Shi J, Wolf S, Burke J, Presting G, {\bf Ross-Ibarra J}, Dawe RK (2010) High frequency gene conversion in centromere cores. PLoS Biology 8: e1000327
%\item Whitney KD, Baack EJ, Hamrick JL, Godt, MJW., Barringer BC, Bennet MD, Eckert CG, Goodwillie C, Kalisz S, Leitch I, {\bf Ross-Ibarra J} (2010) A role for nonadaptive processes in plant genome size evolution? Evolution 64: 2097-2109
%\item van Heerwaarden J, {\bf Ross-Ibarra J}, Doebley J, Glaubitz JC, S\'{a}nchez Gonz\'{a}lez J, Gaut BS, Eguiarte LE (2010) Fine scale genetic structure in the wild ancestor of maize (\emph{Zea mays} ssp. \emph{parviglumis}). Molecular Ecology 19: 1162-1173
%\item Fuchs EJ, {\bf Ross-Ibarra J}, Barrantes G (2010) Reproductive biology of \emph{Macleania rupestris} (Ericaceae): a pollen-limited Neotropical cloud-forest species in Costa Rica. Journal of Tropical Ecology 26: 351-354
%\item van Heerwaarden J, van Eeuwijk FA, {\bf Ross-Ibarra J} (2010) Genetic diversity in a crop metapopulation. Heredity 104: 28-39
%\item Hollister JD, {\bf Ross-Ibarra J}, Gaut BS (2010) Indel-associated mutation rate varies with mating system in flowering plants. Mol Biol Evol 27: 409-416.
%\item {\bf Ross-Ibarra J}, Tenaillon M, Gaut BS (2009) Historical divergence and gene flow in the genus Zea. Genetics 181: 1399-1413.
%\item Gore MA, Chia JM, Elshire RJ, Sun Q, Ersoz ES, Hurwitz BL, Peiffer JA, McMullen MD, Grills GS, {\bf Ross-Ibarra J}, Ware DH, Buckler ES (2009) A first-generation haplotype map of maize. Science 326: 1115-1117.
%\item Zhang LB, Zhu Q, Wu ZQ, {\bf Ross-Ibarra J}, Gaut BS, Ge S, Sang T (2009) Selection on grain shattering genes and rates of rice domestication. New Phytol 184: 708-720.
%\item May MR, Provance MC, Sanders AC, Ellstrand NC, {\bf Ross-Ibarra J} (2009) A pleistocene clone of Palmer's Oak persisting in Southern California. PLoS ONE 4: e8346.
%\item Gaut BS, {\bf Ross-Ibarra J} (2008) Selection on major components of angiosperm genomes. Science 320: 484-486.
%\item {\bf Ross-Ibarra J}, Gaut BS (2008) Multiple domestications do not appear monophyletic. PNAS 105: E105
%\item Lockton S, {\bf Ross-Ibarra J}, Gaut BS (2008) Demography and weak selection drive patterns of transposable element diversity in natural populations of \emph{Arabidopsis lyrata}. PNAS 105: 13965-13970.
%\item {\bf Ross-Ibarra J}, Wright SI, Foxe JP, Kawabe A, DeRose-Wilson L, Gos G, Charlesworth D, Gaut BS (2008) Patterns of polymorphism and demographic history in natural populations of \emph{Arabidopsis lyrata}. PLoS ONE 3: e2411.
%\item {\bf Ross-Ibarra J} (2007) Genome size and recombination in angiosperms: a second look. J Evol Biol 20: 800-806.
%\item {\bf Ross-Ibarra J}, Morrell PL, Gaut BS (2007) Plant domestication, a unique opportunity to identify the genetic basis of adaptation. PNAS 104 Suppl 1: 8641-8648. 
%\item Wares JP, Barber PH, {\bf Ross-Ibarra J}, Sotka EE, Toonen RJ (2006) Mitochondrial DNA and population size. Science 314: 1388-90.
%\item {\bf Ross-Ibarra J} (2005) Quantitative trait loci and the study of plant domestication. Genetica 123: 197-204. 
%\item {\bf Ross-Ibarra J} (2004) The evolution of recombination under domestication: a test of two hypotheses. American Naturalist 163: 105-112.
%\item {\bf Ross-Ibarra J} (2003) Origin and domestication of chaya (\emph{Cnidoscolus aconitifolius} Mill I. M. Johnst): Mayan spinach. Mexican Studies 19: 287-302.
%\item {\bf Ross-Ibarra J}, Molina-Cruz A (2002) The ethnobotany of Chaya (\emph{Cnidoscolus aconitifolius} ssp. \emph{aconitifolius} Breckon): A nutritious Maya vegetable. Economic Botany 56: 350-365.
%\item  Neel MC, {\bf Ross-Ibarra J}, Ellstrand NC (2001) Implications of mating patterns for conservation of the endangered plant \emph{Eriogonum ovalifolium} var. \emph{vineum} (Polygonaceae). American J Botany 88: 1214-1222.

\end{etaremune}

%BOOK CHAPTERS
%\subsection*{Book Chapters}
%\begin{itemize}
%\item Ross-Ibarra, J, PL Morrell and BS Gaut 2007 Plant Domestication, a Unique Opportunity to Identify the Genetic Basis of Adaptation In: In the Light of Evolution Editors: FJ Ayala and J Avise The National Academies Press Washington, DC 
%\item Ross-Ibarra, J 2005 QTL Mapping and the study of plant domestication In: Genetics of Adaptation Springer, Dordrecht 
%\end{itemize}

\bigskip

% Footer
\begin{center}
  \begin{footnotesize}
    Last updated: \today \\
    \href{\footerlink}{\texttt{\footerlink}}
  \end{footnotesize}
\end{center}

\end{document}